\hypertarget{group__WeakLinkage}{\section{managing support for newer/older versions of \-J\-A\-C\-K}
\label{d9/dcc/group__WeakLinkage}\index{managing support for newer/older versions of J\-A\-C\-K@{managing support for newer/older versions of J\-A\-C\-K}}
}
\-One challenge faced by developers is that of taking advantage of new features introduced in new versions of \mbox{[} \-J\-A\-C\-K \mbox{]} while still supporting older versions of the system. \-Normally, if an application uses a new feature in a library/\-A\-P\-I, it is unable to run on earlier versions of the library/\-A\-P\-I that do not support that feature. \-Such applications would either fail to launch or crash when an attempt to use the feature was made. \-This problem cane be solved using weakly-\/linked symbols.

\-When a symbol in a framework is defined as weakly linked, the symbol does not have to be present at runtime for a process to continue running. \-The static linker identifies a weakly linked symbol as such in any code module that references the symbol. \-The dynamic linker uses this same information at runtime to determine whether a process can continue running. \-If a weakly linked symbol is not present in the framework, the code module can continue to run as long as it does not reference the symbol. \-However, if the symbol is present, the code can use it normally.

(adapted from\-: \href{http://developer.apple.com/library/mac/#documentation/MacOSX/Conceptual/BPFrameworks/Concepts/WeakLinking.html}{\tt http\-://developer.\-apple.\-com/library/mac/\#documentation/\-Mac\-O\-S\-X/\-Conceptual/\-B\-P\-Frameworks/\-Concepts/\-Weak\-Linking.\-html})

\-A concrete example will help. \-Suppose that someone uses a version of a \-J\-A\-C\-K client we'll call \char`\"{}\-Jill\char`\"{}. \-Jill was linked against a version of \-J\-A\-C\-K that contains a newer part of the \-A\-P\-I (say, \hyperlink{jack_8h_a70a38fb1e74c5e9df9f1305c695c58bf}{jack\-\_\-set\-\_\-latency\-\_\-callback()}) and would like to use it if it is available.

\-When \-Jill is run on a system that has a suitably \char`\"{}new\char`\"{} version of \-J\-A\-C\-K, this function will be available entirely normally. \-But if \-Jill is run on a system with an old version of \-J\-A\-C\-K, the function isn't available.

\-With normal symbol linkage, this would create a startup error whenever someone tries to run \-Jill with the \char`\"{}old\char`\"{} version of \-J\-A\-C\-K. \-However, functions added to \-J\-A\-C\-K after version 0.\-116.\-2 are all declared to have \char`\"{}weak\char`\"{} linkage which means that their abscence doesn't cause an error during program startup. \-Instead, \-Jill can test whether or not the symbol jack\-\_\-set\-\_\-latency\-\_\-callback is null or not. \-If its null, it means that the \-J\-A\-C\-K installed on this machine is too old to support this function. \-If its not null, then \-Jill can use it just like any other function in the \-A\-P\-I. \-For example\-:


\begin{DoxyCode}
 if (jack_set_latency_callback) {
       jack_set_latency_callback (jill_client, jill_latency_callback, arg);
 }
\end{DoxyCode}


\-However, there are clients that may want to use this approach to parts of the the \-J\-A\-C\-K \-A\-P\-I that predate 0.\-116.\-2. \-For example, they might want to see if even really old basic parts of the \-A\-P\-I like \hyperlink{jack_8h_abbd2041bca191943b6ef29a991a131c5}{jack\-\_\-client\-\_\-open()} exist at runtime.

\-Such clients should include $<$\hyperlink{weakjack_8h}{jack/weakjack.\-h}$>$ before any other \-J\-A\-C\-K header. \-This will make the {\bfseries entire} \-J\-A\-C\-K \-A\-P\-I be subject to weak linkage, so that any and all functions can be checked for existence at runtime. \-It is important to understand that very few clients need to do this -\/ if you use this feature you should have a clear reason to do so. 